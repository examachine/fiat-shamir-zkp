% -*-latex-*-
% Typical article template

\documentclass[a4paper,12pt]{article}
\usepackage{amsmath}
\usepackage{amstext}
\usepackage{amsbsy}
\title{An implementation of ZKPS authentication scheme of Fiat-Shamir }

\author{Eray \"Ozkural}

\date{\today}

\begin{document}

\maketitle

This program implements a cryptographic network authentication scheme
based on the work of Fiat-Shamir. The system is described in Section
6.8 of Network Security: Private Communication in a Public World by
Kaufman et al. Originally, I intended to implement the graph
isomorphism based zero knowledge proof system. However, this system is
regarded a little inefficient.

Generally, the advantage of a zero knowledge proof system is that one
can prove he knows a secret without revealing it. As one may have
noticed, this is also the basic idea in public-key cryptography where
the computational load of not knowing the secret determines the
strength of the crypto-system. The Fiat-Shamir scheme is essentially a
public-key cryptographic authentication scheme. Alice chooses a public
key and a secret key, and Alice can prove her identity only through
her public key she shares with Bob. No other information is required
to achieve the authentication and an eavesdropper cannot replay an
authentication session.

Although it is described in detail in the textbook, I will describe
the scheme once again here for sake of completeness. Just like in RSA,
Alice first chooses two large random primes $p$ and $q$. Let $n=pq$.
Alice also picks a secret $s$. Let $v=s^2\ \text{mod}\ n$. Alice's public key
is $(n,v)$ and her secret key is $s$. The factors $p$ and $q$ can be
deleted after key computation.

The protocol is very simple in essence and as we will see has a
challenge-response conversation typical in zero knowledge proof
systems. Here is the protocol initiated by Alice to prove her identity to Bob
:

\begin{enumerate}
\item Alice chooses $k$ random numbers $r_1,r_2,...,r_k$. For each
$r_i$, she sends $r_i^2 \ \text{mod}\  n$ to Bob

\item Bob chooses a random subset of the $r_i^2$ and tells Alice
  which subset he has selected to be known as subset 1. The
  others will be known as subset 2

\item Alice sends $sr_i\  \text{mod}\  n$ for each $r_i^2$ of subset 1, and
  sends $r_i\  \text{mod}\  n$ for each $r_i^2$ of subset 2
\item Bob squares Alice's replies $\ \text{mod}\  n$. For those $r_i^2$ in
  subset 1 he checks that the square of the reply is $vr_i^2
  \ \text{mod}\  n$. For those $r_i^2$ in subset 2, he checks that the
  square of the reply is $ r_i^2\  \text{mod}\  n$.

\end{enumerate}

The protocol's security depends on the fact that square root mod $n$
is at least as hard as factoring $n$. That is, given an efficient
square root mod $n$ algorithm one can factor $n$ efficiently. The
problem transformation is provided in the textbook.

Let Trudy be an eavesdropper who wishes to impersonate Alice. By
overhearing the conversation she will have collected pairs $<r_i^2,
sr_i>$. However, as can be verified from the protocol, she cannot have
$r_i$. Therefore, she cannot at the same time compute both $sr_i mod
n$ and $r_i mod n$ for Bob chooses the partition of random numbers,
randomly. Trudy's chance in getting the answer right is $50 \%$ for
each challenge. For this reason the probability that it she will
succeed in impersonation is only $2^{-k}$, which is enormously low for
typical values of $k$. In the textbook a value of $30$ is being given
as example.

The Fiat-Shamir scheme has the added advantage of being as secure as
RSA, while requiring much less modular operations on big numbers. A
comparison of typical scenarios in the textbook show that the work for
Alice is about the same while the work for Alice is hundred times
less.

The implementation was done in the high level efficient functional
language ocaml. I've used bits of cryptokit by Xavier Leroy to
implement such things as generating random primes. The authentication
occurs over the network. The client and server program work 
not unlike $ssh$ or $gpg$. I haven't built an application on the
authentication scheme. However, the UNIX CLIs allow for key generation
and then authentication over TCP/IP.

\end{document}
